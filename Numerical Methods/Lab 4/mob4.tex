\documentclass[a4paper,14pt]{extarticle} % the default article class is limited to 12pt, but you can go up to 14, 17 or 20 points if you use the extarticle class:
\usepackage{cmap} % make LaTeX PDF output copy-and-pasteable
\usepackage[T2A]{fontenc}
\usepackage[utf8]{inputenc}
\usepackage[english,ukrainian]{babel}

\usepackage{amssymb,amsfonts,mathtools,amsmath,cite,enumerate,float}
\usepackage{indentfirst} % set an additional space before a paragraph at the begining of new section
\usepackage{setspace}
\usepackage{textcomp}

\usepackage{leftidx}

\usepackage{import} % for adding a file by path https://tex.stackexchange.com/questions/246/when-should-i-use-input-vs-include

\usepackage{geometry} 
\geometry{left=1.25cm}
\geometry{right=1.25cm}
\geometry{top=1cm}
\geometry{bottom=2cm}

\usepackage[table,xcdraw,dvipsnames]{xcolor}
\usepackage{color}
% 1) tutorial about xcolor:  https://www.overleaf.com/learn/latex/Using_colours_in_LaTeX
% 2) huge tutorial about xcolor: https://latex-tutorial.com/color-latex/ 
% 3) RGB calculator: https://www.w3schools.com/colors/colors_rgb.asp

\usepackage{hyperref}
\definecolor{linkcolor}{HTML}{0000FF}
\definecolor{urlcolor}{HTML}{0000FF} 
\hypersetup{pdfstartview=FitH, linkcolor=linkcolor, urlcolor=urlcolor, colorlinks=true}

\usepackage{graphicx}
\usepackage{wrapfig}
\graphicspath{{Screenshots/}} % path to images

\parskip=1mm %space between paragraphs

\usepackage{listingsutf8} % for code (origin: \usepackage{listings})

\lstset{
    frame=single, %lines
    language=Python,
    aboveskip=3mm,
    belowskip=3mm,
    columns=flexible,
    basicstyle={\small\ttfamily},
    numbers=left,
    numberstyle=\tiny\color{gray},
    commentstyle=\color{OliveGreen},
    stringstyle=\color{Mahogany},
    morestring=[b]''',
    showstringspaces=false,
    keywordstyle=\bfseries\color{blue},
    emph={[1]import, as, for, return}, emphstyle={[1]\bfseries\color{magenta}},
    emph={[2]range}, emphstyle={[2]\bfseries\color{brown}},
    breaklines=true,
    breakatwhitespace=true,
    tabsize=4,
    extendedchars=false, % to use ukrainian text in a code
    inputencoding=utf8 % to use ukrainian text in a code
}

\begin{document}

\import{Title/}{title}

\newpage

\subsection*{Завдання}

Завдання лабораторної роботи полягає у пошуку власних чисел деякої заданої матриці $A$. До цього ми 
знайомилися із способом знаходження цих чисел з рівняння $\det(A-\lambda I)=0$, що є обчислювано нераціональним. 
Тож наразі виконаємо необхідні розрахунки методом Якобі: приведемо вихідну матрицю за допомогою подібних 
обертань до майже діагонального вигляду (сума модулів недіагональних елементів має дорівнювати нулю із точністю 
$\varepsilon=10^{-5}$), і тоді на головній діагоналі будуть міститись наближені значення власних чисел.

\subsection*{Варіант завдання}

Задана така симетрична матриця $A:$
\[
A=
    \begin{pmatrix}
        7 & 0.88 & 0.93 & 1.21 \\
        0.88 & 4.16 & 1.3 & 0.15 \\
        0.93 & 1.3 & 6.44 & 2 \\
        1.21 & 0.15 & 2 & 6.1 \\
    \end{pmatrix}
\]

\subsection*{Теоретичні відомості}

Метод Якобі діє для симетричних матриць: ітераційно виконуватимемо процес обертання, починаючи з матриці $A$, й
пильнуватимемо момент, коли виконуватиметься умова завершення. Обертанням будемо називати перетворення координат за допомогою матриці 
$T_{ij}$ виду:
\[
T_{ij}=
    \begin{pmatrix}
        1 & 0 & 0 & \ldots & 0 & 0 & 0\\
        0 & 1 & 0 & \ldots & 0 & 0 & 0\\
        0 & 0 & t_{ii} & \ldots & t_{ji} & 0 & 0 \\
        \vdots & \vdots & \vdots & & \vdots & \vdots & \vdots \\
        0 & 0 & t_{ij} & \ldots & t_{jj} & 0 & 0 \\
        0 & 0 & 0 & \ldots & 0 & 1 & 0 \\
        0 & 0 & 0 & \ldots & 0 & 0 & 1 \\
    \end{pmatrix},\ \text{де} \
    \begin{cases}
        t_{ii} = t_{jj} = c \\
        t_{ij} = \text{-}s,\ t_{ji} = s \\
    \end{cases}
\]

При цьому для цієї матриці її обернена рівна транспонованій: $T^T=T^{-1}$. Числа $c$ та $s$ знаходимо із таких умов:
\[ \mu=\frac{2a_{ij}}{a_{ii}-a_{jj}},\ c=\sqrt{\frac{1}{2}\left(1+\frac{1}{\sqrt{1+\mu^2}}\right)},
\ s=sign(\mu)\sqrt{\frac{1}{2}\left(1-\frac{1}{\sqrt{1+\mu^2}}\right)} \]

Відповідно основна ітерація методу виглядатиме як перетворення виду 
\[ A_{k+1}=T_{ij}^TA_{k}T_{ij} \]

доти, доки $A_{k+1}$-ша матриця не буде задовільняти умові майже діагонального вигляду (сума модулів 
недiагональних елементiв майже рівна нулю).

\newpage
Залишається питання ідексів $i,j$: для кожної ітерації ці індекси визначатимуться як індекси максимального по 
модулю недіагонального елемента поточної матриці $A_{k}$, при цьому сам елемент $a_{ij}$ кожного разу слід 
обнуляти. Для моніторингу поточної ситуації на кожній ітерації введемо величини:
\begin{align*}
    &S_A=\sum\limits_{i,j=1}^n a_{ij}^2 && \text{сферична норма матриці} \\
    &S_d=\sum\limits_{i=1}^n a_{ii}^2 && \text{діагональна частина сферичної норми} \\
    &S_{nd}=\sum\limits_{i,j=1,i\neq j}^n a_{ij}^2 && \text{недіагональна частина сферичної норми}
\end{align*}

Тож від ітерації до ітерації $S_A$ має лишатися сталою, $S_{nd}$ -- зменшуватися, а $S_{d}$, відповідно, збільшуватися.

\subsection*{Програмний код}

\lstinputlisting{Code/Jacobi.py}

\newpage
\subsection*{Результати й проміжні кроки}

\begin{lstlisting}[language=bash, stringstyle=\small\ttfamily, emphstyle={[1]\small\ttfamily}, frame=none, numbers=none]
Iteration 1:

    Matrix T
    [[ 1.          0.          0.          0.        ]
     [ 0.          1.          0.          0.        ]
     [ 0.          0.          0.87722679  0.48007619]
     [ 0.          0.         -0.48007619  0.87722679]]
    
    Matrix T'
    [[ 1.          0.          0.          0.        ]
     [ 0.          1.          0.          0.        ]
     [ 0.          0.          0.87722679 -0.48007619]
     [ 0.          0.          0.48007619  0.87722679]]
    
    Sa = 206.41099999999997, Snd = 9.631799999999998, Sd = 196.7792
    Value of S = 8.8938209584238
    
Iteration 2:
    
    Matrix T
    [[ 0.92632469  0.          0.          0.37672612]
     [ 0.          1.          0.          0.        ]
     [ 0.          0.          1.          0.        ]
     [-0.37672612  0.          0.          0.92632469]]
    
    Matrix T'
    [[ 0.92632469  0.          0.         -0.37672612]
     [ 0.          1.          0.          0.        ]
     [ 0.          0.          1.          0.        ]
     [ 0.37672612  0.          0.          0.92632469]]
    
    Sa = 206.411, Snd = 5.084183010774322, Sd = 201.32681698922568
    Value of S = 5.8730290283892534
    
Iteration 3:
    
    Matrix T
    [[ 1.          0.          0.          0.        ]
     [ 0.          0.86172354  0.5073781   0.        ]
     [ 0.         -0.5073781   0.86172354  0.        ]
     [ 0.          0.          0.          1.        ]]
    
    Matrix T'
    [[ 1.          0.          0.          0.        ]
     [ 0.          0.86172354 -0.5073781   0.        ]
     [ 0.          0.5073781   0.86172354  0.        ]
     [ 0.          0.          0.          1.        ]]
    
    Sa = 206.411, Snd = 2.801296818246346, Sd = 203.60970318175364
    Value of S = 4.494040159643841
    
Iteration 4:
    
    Matrix T
    [[ 1.          0.          0.          0.        ]
     [ 0.          0.99333736  0.          0.11524272]
     [ 0.          0.          1.          0.        ]
     [ 0.         -0.11524272  0.          0.99333736]]
    
    Matrix T'
    [[ 1.          0.          0.          0.        ]
     [ 0.          0.99333736  0.         -0.11524272]
     [ 0.          0.          1.          0.        ]
     [ 0.          0.11524272  0.          0.99333736]]
    
    Sa = 206.411, Snd = 1.3766717130506674, Sd = 205.03432828694932
    Value of S = 3.0115777795057697
    
Iteration 5:
    
    Matrix T
    [[ 1.          0.          0.          0.        ]
     [ 0.          1.          0.          0.        ]
     [ 0.          0.          0.99270285  0.1205863 ]
     [ 0.          0.         -0.1205863   0.99270285]]
    
    Matrix T'
    [[ 1.          0.          0.          0.        ]
     [ 0.          1.          0.          0.        ]
     [ 0.          0.          0.99270285 -0.1205863 ]
     [ 0.          0.          0.1205863   0.99270285]]
    
    Sa = 206.411, Snd = 0.6670863655113755, Sd = 205.74391363448862
    Value of S = 1.92920070120573
    
Iteration 6:
    
    Matrix T
    [[ 0.85885707  0.         -0.51221532  0.        ]
     [ 0.          1.          0.          0.        ]
     [ 0.51221532  0.          0.85885707  0.        ]
     [ 0.          0.          0.          1.        ]]
    
    Matrix T'
    [[ 0.85885707  0.          0.51221532  0.        ]
     [ 0.          1.          0.          0.        ]
     [-0.51221532  0.          0.85885707  0.        ]
     [ 0.          0.          0.          1.        ]]
    
    Sa = 206.411, Snd = 0.26472461148438275, Sd = 206.14627538851562
    Value of S = 1.268398439619539

\end{lstlisting}

\newpage
\begin{lstlisting}[language=bash, stringstyle=\small\ttfamily, emphstyle={[1]\small\ttfamily}, frame=none, numbers=none]
Iteration 7:
    
    Matrix T
    [[ 0.99678393 -0.08013611  0.          0.        ]
     [ 0.08013611  0.99678393  0.          0.        ]
     [ 0.          0.          1.          0.        ]
     [ 0.          0.          0.          1.        ]]
    
    Matrix T'
    [[ 0.99678393  0.08013611  0.          0.        ]
     [-0.08013611  0.99678393  0.          0.        ]
     [ 0.          0.          1.          0.        ]
     [ 0.          0.          0.          1.        ]]
    
    Sa = 206.411, Snd = 0.12884386678223536, Sd = 206.28215613321774
    Value of S = 0.7944211503429432
    
Iteration 8:
    
    Matrix T
    [[ 1.          0.          0.          0.        ]
     [ 0.          0.99459572 -0.10382364  0.        ]
     [ 0.          0.10382364  0.99459572  0.        ]
     [ 0.          0.          0.          1.        ]]
    
    Matrix T'
    [[ 1.          0.          0.          0.        ]
     [ 0.          0.99459572  0.10382364  0.        ]
     [ 0.         -0.10382364  0.99459572  0.        ]
     [ 0.          0.          0.          1.        ]]
    
    Sa = 206.411, Snd = 0.018799217402917574, Sd = 206.39220078259706
    Value of S = 0.335308936475719
\end{lstlisting}

Наведено результати й проміжні кроки для перших $k=8$ ітерацій. Назагал, бажана точність на суму модулів недіагональних елементів 
досягнута при $k=15$ ітераціях, і отриманий результат рівний:

\[
A^*=
    \begin{pmatrix}
        6.67397938 & 0 & 0 & 0 \\
        0 & 3.38753165 & 0 & 0 \\
        0 & 0 & 5.65841896 & 0 \\
        0 & 0 & 0 & 10.88007000 \\
    \end{pmatrix}
\]

Таким чином знайдені власні числа рівні: 
\[ \lambda_1=6.67397938,\  \lambda_2=3.38753165,\ \lambda_3=5.65841896,\ \lambda_4=10.88007 \]

Скориставшись онлайн сайтами знаходження власних чисел, матимемо аналогічні результати.

\subsection*{Контрольні запитання}

\begin{enumerate}
    \item \textit{Які рядки та стовпці матриці $A^{k+1}$ змінює процес обертання $T_{ij}$ порівняно із 
    вихідною матрицею $A$ в методі Якобі?}

    Якщо над матрицею $A^{k}$ провести процес обертання $A^{k+1}=T_{ij}^T A^k T_{ij}\equiv T_{ij}^T B^k$, то 
    лише рядки та стовпці $i$ та $j$ відповідних матриць будуть змінюватися таким чином:
    \begin{align*}
        & B^k_i=cA^k_i+sA^k_j && A^{k+1}_i=cB^k_i+sB^k_j \\
        & B^k_j=-sA^k_i+cA^k_j && A^{k+1}_j=-sB^k_i+cB^k_j   
    \end{align*}

    \item \textit{Як за методом Якобі визначити власні вектори матриці $A$ за результуючою матрицею $A^*$,
    знаючи сукупність перетворень $\{T_{ij}\}$?}

    Довільну симетричну матрицю $A$ можна представити у вигляді $A=TA^*T^{-1}$, де $A^*$ -- матриця, складена 
    з власних значень, а $T$ -- матриця, що складається із власних векторів. 
    
    У методі Якобі, провівши, скажімо, $n$ штук ітерації виду $A_{k+1}=T_{ij}^TA_{k}T_{ij}$ над 
    початковою матрицею $A$ віднайшли шукану матрицю власних чисел $A^*$. Тобто з одного боку
    \[ A^*= \underbracket{{^n}T_{ij}^T\cdot {^{n-1}}T_{ij}^T\ldots \cdot {^1}T_{ij}^T} \cdot A \cdot \underbracket{{^1}T_{ij}\cdot \ldots {^{n-1}}T_{ij}\cdot {^n}T_{ij}} \] 
    а з іншого
    \[ A=TA^*T^{-1} \Rightarrow A^*=T^{-1}AT \]

    Отже, порівнявши ці два представлення матимемо, що шукана матриця власних векторів $T$ із відомою 
    сукупністю перетворень $\{T_{ij}\}$ знаходитиметься так:
    \[ T={^1}T_{ij}\cdot \ldots {^{n-1}}T_{ij}\cdot {^n}T_{ij} \]

    \item \textit{Які елементи матриці будуть зменшуватися при обертаннях за методом Якобі, а які будуть
    збільшуватись?}

    Як вже зазначалося, ввівши величини
    \[ S_A=\sum\limits_{i,j=1}^n a_{ij}^2,\ S_d=\sum\limits_{i=1}^n a_{ii}^2,\ S_{nd}=\sum\limits_{i,j=1,i\neq j}^n a_{ij}^2 \]
    
    від ітерації до ітерації $S_A$ має лишатися сталою, сума квадратів недіагональних елементів $S_{nd}$ -- 
    зменшуватися, а сума квадратів діагональних елементів $S_{d}$, відповідно, збільшуватися.
    
    \item \textit{Коли метод Данилевського неможливо застосувати?}

    Вказаний метод неможливо застосувати тоді, коли при формуванні матриць $M_k$ виникає ділення на нульові 
    коефіцієнти поточної матриці $A_k$. 
    
    \item \textit{У яких випадках система характеристичного поліному із коефіцієнтами $p_{i}$ буде виродженою?}
    
    Система характеристичного поліному із коефіцієнтами $p_{i}$ буде виродженою тоді, коли матриця $A$ має 
    кратні класні власні числа. Тобто у такому разі зазначена система є лінійно залежною, що призводить до обнулення 
    визначника матриці коефіцієнтів системи.
\end{enumerate}

\end{document}