\documentclass[a4paper,14pt]{extarticle} %the default article class is limited to 12pt, but you can go up to 14, 17 or 20 points if you use the extarticle class:
\usepackage{cmap} %make LaTeX PDF output copy-and-pasteable
\usepackage[T2A]{fontenc}
\usepackage[utf8]{inputenc}
\usepackage[english,ukrainian]{babel}

\usepackage{amssymb,amsfonts,amsmath,cite,enumerate,float}
\usepackage{indentfirst} %set an additional space before a paragraph at the begining of new section
\usepackage{setspace}
\usepackage{textcomp}

\usepackage{geometry} 
\geometry{left=1.25cm}
\geometry{right=1.25cm}
\geometry{top=1cm}
\geometry{bottom=2cm}

\usepackage{graphicx}
\usepackage{wrapfig}
\graphicspath{{images/}} %path to images

\parskip=1mm %space between paragraphs

\usepackage[dvipsnames]{xcolor}
\usepackage{color}
% 1) tutorial about xcolor:  https://www.overleaf.com/learn/latex/Using_colours_in_LaTeX
% 2) huge tutorial about xcolor: https://latex-tutorial.com/color-latex/ 
% 3) RGB calculator: https://www.w3schools.com/colors/colors_rgb.asp

\usepackage{minted} 
\usemintedstyle[Python]{vs}
% 1) minted (installing): https://leportella.com/minted-vscode/
% 2) minted (styles): https://sharelatex.psi.ch/learn/Code_Highlighting_with_minted
% 3) minted (environment options): https://latex-tutorial.com/code-listings/

\usepackage{listings} %for code

\lstset{
    frame=single, %lines
    language=Python,
    aboveskip=3mm,
    belowskip=3mm,
    columns=flexible,
    basicstyle={\small\ttfamily},
    numbers=left,
    numberstyle=\tiny\color{gray},
    commentstyle=\color{OliveGreen},
    stringstyle=\color{mauve},
    keywordstyle=\bfseries\color{Blue},
    emph={[1]import, as, for, return}, emphstyle={[1]\color{violet}},
    emph={[2]range, abs, build_t, build_t_rev, find_x, find_y}, emphstyle={[2]\color{Brown}},
    breaklines=true,
    breakatwhitespace=true,
    tabsize=4
}

\begin{document}

\begin{titlepage}
    \newpage

    \begin{minipage}[c]{\linewidth}

        \newlength{\maxpreambula}
        \settowidth{\maxpreambula}{\small{<<Київський політехнічний інститут імені Ігоря Сікорського>>}}    

        \hspace{5cm}\parbox{\maxpreambula}{
            \begin{spacing}{1.1}\small{
                Міністерство освіти і науки України \\
                Національний технічний університет України \\
                <<Київський політехнічний інститут імені Ігоря Сікорського>> \\
                Фізико-технічний інститут }
            \end{spacing}
        }
            
        \vspace*{-2.35cm}
        \hspace*{1.5cm}
        \includegraphics[width=0.13\paperwidth]{kpi_emblem.png}

    \end{minipage}
    
    \vspace{13em}
    
    \begin{center}
        \begin{spacing}{1.5}
            \textbf{\Large{Комп'ютерний практикум № 2}} \\
            \vspace{1cm}\textbf{\Large{Побудова графіка функції \linebreak 
            за алгоритмом Брезенхейма}} \\
            \vspace{1cm}\textbf{\large{предмет <<Комп'ютерна графіка>>}}
        \end{spacing}
    \end{center}

    \vspace{13em}

    \newlength{\maxname}
    \settowidth{\maxname}{\small{Студент 3 курсу ФТІ, групи ФІ-91}}

    \hfill\parbox{\maxname}{
        \begin{spacing}{1.1}
            \small{\textbf{Роботу виконав:}} \\ 
            \small{Студент 3 курсу ФТІ, групи ФІ-91} \\
            \small{Цибульник Антон Владиславович} \\
        \end{spacing}
    }

    \hfill\parbox{\maxname}{
        \begin{spacing}{1.1}
            \small{\textbf{Приймав:}} \\ 
            \small{Професор кафедри ІБ} \\
            \small{Півень Олег Борисович} \\
        \end{spacing}
    }

    \vspace{\fill}
    
    \begin{center}
        \small{2022}
    \end{center}
    
\end{titlepage}

\newpage

\subsection*{Завдання}

Розв’язати систему рівнянь з кількістю значущих цифр $m=6$. Якшо матриця системи
симетрична, то розв’язання проводити за методом квадратних коренів, якщо матриця системи
несиметрична, то використати метод Гауса. Вивести всі проміжні результати (матриці $A$, що
отримуються в ході прямого ходу методу Гауса, матрицю зворотного ходу методу Гауса, або
матрицю $T$ та вектор $y$ для методу квадратних коренів), та розв’язок системи. Навести 
результат перевірки: вектор нев'язки $r = |b - Ax|$, де $x$ -- отриманий розв’язок.

\subsection*{Варіант завдання}

Розглядається така система рівнянь:

\begin{equation}
    \left\{
    \begin{aligned}
        &5{,}5x_1 + 7x_2 + 6x_3 + 5{,}5x_4 = 23 \\
        &7x_1 + 10{,}5x_2 + 8x_3 + 7x_4 = 32 \\
        &6x_1 + 8x_2 + 10{,}5x_3 + 9x_4 = 33 \\
        &5{,}5x_1 + 7x_2 + 9x_3 + 10{,}5x_4 = 31 \\
    \end{aligned} \label{equations}
    \right.
\end{equation}

З цієї системи запишемо матрицю коефіцієнтів $A$ та матрицю вільних членів $B$:

\[
A=
    \begin{pmatrix}
        5.5 & 7 & 6 & 5.5 \\
        7 & 10.5 & 8 & 7 \\
        6 & 8 & 10.5 & 9 \\
        5.5 & 7 & 9 & 10.5 \\
    \end{pmatrix},
B = 
    \begin{pmatrix}
        23 \\
        32 \\
        33 \\
        31 \\
    \end{pmatrix}.
\]

Матриця $A$ є симетричною, отже, завдання зводиться до 
знаходження розв'язку системи рівнянь методом квадратного кореня.

\subsection*{Теоретичні відомості}

Метод квадратного кореня полягає у тому, щоб спершу представити 
початкову матрицю $A$ у вигляді добутку двох взаємно транспонованих матриць $A=TT'$, 
а далі зворотнім ходом віднайти шуканий розв'язок $x$ початкової системи рівнянь. 
Тож на першому кроці відбувається так звана факторизація, тобто прямий хід, 
а на другому кроці -- зворотній хід, тобто фінальне знаходження коренів. \\

\textbf{Крок 1.} Прямий хід (факторизація):

\begin{equation}
\label{formula:TT'}
    \begin{pmatrix}
        t_{11} & 0 & 0 & 0 \\
        t_{21} & t_{22} & 0 & 0 \\
        t_{31} & t_{32} & t_{33} & 0 \\
        t_{41} & t_{42} & t_{43} & t_{44} \\
    \end{pmatrix}
    \begin{pmatrix}
        t_{11} & t_{12} & t_{13} & t_{14} \\
        0 & t_{22} & t_{23} & t_{24} \\
        0 & 0 & t_{33} & t_{34} \\
        0 & 0 & 0 & t_{44} \\
    \end{pmatrix} = 
    \begin{pmatrix}
        a_{11} & a_{12} & a_{13} & a_{14} \\
        a_{21} & a_{22} & a_{23} & a_{24} \\
        a_{31} & a_{32} & a_{33} & a_{34} \\
        a_{41} & a_{42} & a_{43} & a_{44} \\
    \end{pmatrix},
\end{equation}

\newpage
де елементи матриці $T$ шукатимемо за співвідношеннями, 
які можна отримати \par послідовним перемноженням елементів у рівнянні 
(\ref{formula:TT'}):

\begin{align*}
    t_{11}=\sqrt{a_{11}}\ , t_{i1}=\frac{a_{i1}}{t_{11}}& &&\text{для елементів першого стовпця,} \\ 
    t_{ii}=\sqrt{a_{ii}-\sum\limits_{k<i}t_{ik}^2}& &&\text{для діагональних елементів,} \\
    t_{ij}=\frac{a_{ij}-\sum\limits_{k<j}t_{ik}t_{jk}}{t_{jj}}& &&\text{для елементів під діагоналлю,}
\end{align*}

причому важливим є порядок застосування формул. \\

\begin{wrapfigure}[6]{r}{0.34\linewidth}
    \vspace{-1.1cm}
    \center{\includegraphics[width=1\linewidth]{Images/image.png}}
    \label{fig:instance}
\end{wrapfigure}

Правило обрахунку елементів матриці $T$ можна описати так: спочатку з'ясовуємо перший 
діагональний елемент $a_{11}$, далі обраховуємо всі елементи під ним, тобто елементи 
відповідного стовпця. Наступним кроком рахуємо другий діагональний елемент, а тоді знову всі 
елементи відповідного стовпця і так далі. На рисунку праворуч зображена відповідна схема. \\

\textbf{Крок 2.} Зворотній хід:

\begin{align}
    \text{Із початкової системи рівнянь} \qquad &Ax=B, \\
    \text{тоді враховуючи, що } A=TT' \qquad &TT'x=B, \\
    \parbox{6.9cm}{розіб'ємо отримане рівняння на такі два послідовних підкроки} \qquad 
    &\begin{cases}
        T'x=y \\
        Ty=B. \\
    \end{cases} \label{formula:step 2}
\end{align}

Оскільки кожна з матриць $T$ та $T'$ має діагональний вид, то обидва 
матричні рівняння системи (\ref{formula:step 2}) розв'язуються зворотнім 
ходом (як в методі Гауса). Наведемо формули обрахунку:  

\begin{align*}
    y_1=\frac{b_1}{t_{11}},\ &y_i=\frac{b_i-\sum\limits_{k=1}^{i-1}t_{ik}y_k}{t_{ii}} \\
    x_n=\frac{y_n}{t'_{nn}},\ &x_i=\frac{y_i-\sum\limits_{k=i+1}^{n}t'_{ik}x_k}{t'_{ii}}. \\
\end{align*}

\subsection*{Програмний код}

\lstinputlisting{mkk.py}

\newpage
\subsection*{Результати й проміжні кроки}

Наведемо проміжні значення матриць $T, T'$ методу квадратного кореня:

\begin{align}
T = 
\begin{pmatrix} 
    2.34520788 & 0          & 0          & 0          \\
    2.98481003 & 1.26131245 & 0          & 0          \\
    2.5584086  & 0.28829998 & 1.96759461 & 0          \\
    2.34520788 & 0          & 1.52470431 & 1.63562733 \\
\end{pmatrix} \\ {} \notag \\
T' =
\begin{pmatrix}
    2.34520788 & 2.98481003 & 2.5584086  & 2.34520788 \\
    0          & 1.26131245 & 0.28829998 & 0          \\
    0          & 0          & 1.96759461 & 1.52470431 \\
    0          & 0          & 0          & 1.63562733 \\
\end{pmatrix}
\end{align}

Визначити правильність зазначених матриць можна, власне кажучи, 
зробивши таку перевірку: $A=TT'$. Також варто зазначити проміжний вектор $y$, 
отриманий після розв'язку зворотнім ходом рівняння $Ty=B$:
\[ y = (9.80723295, 2.16224991, 3.70285334, 1.43935205). \]

І наостанок був обрахований вектор нев'язки $r=|B-Ax|$, 
який вказує похибку обчислень (коротше кажучи, це перевірка правильності розв'язку):
\[ r =  (0, 0, 0, 0). \]

Судячи з вектора нев'язки $r$, отриманий розв'язок 
$x=(0.16, 1.44, 1.2, 0.88)$ і є розв'язком початкової 
системи рівнянь $Ax=B$, зображеної у виразі (\ref{equations}).
Отже, остаточний розв'язок: 
\[ x_1=0.16,\ x_2=1.44,\ x_3=1.2,\ x_4=0.88. \]

\subsection*{Контрольні запитання}

\begin{enumerate}
    \item \textit{В чому полягає відмінність схеми Гауса із вибором головного елементу та схеми єдиного ділення?}

    Фактично, відмінність між цими двома методами полягає у виборі головного елементу. Наприклад, у 
    схемі єдиного ділення головним елементом завжди обирають перший коефіцієнт, натомість у методі 
    вибору елементу можемо назначити головний елемент власноруч (наприклад, обрати деякий максимальний). 
    Це запобігає накопиченню похибок округлення. Крім того, це дає можливість уникати ситуацій, 
    коли першим коефіцієнтом є нуль.

    \newpage \item \textit{В чому переваги схеми Гауса із вибором головного елементу порівняно із 
    схемою єдиного ділення, схемою повного виключення?}

    Переваги схеми повного виключення порівняно зі схемою єдиного ділення аналогічні до попереднього питання.

    \item \textit{Коли факторизація за методом квадратного кореня неможлива для симетричної матриці?}

    Відповідь на питання можна визначити із співвідношень, наведених на початку першого кроку теоретичних 
    відомостей. Власне, ситуація факторизації можлива для комплексних значень, проте неможлива у випадку 
    утворення нульових величин на місці діагональних елементів матриць $T$ та $T'$.
\end{enumerate}

\end{document}