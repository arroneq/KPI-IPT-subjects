%!TEX root = ../thesis.tex

\chapter{(Назва першого розділу)}
\label{chap:review}  %% відмічайте кожен розділ певною міткою -- на неї наприкінці необхідно посилатись

На початку кожного розділу рекомендується вставити одне-два-абзац речень, 
у яких коротенько представили, про що тут взагалі буде мова.

Звіт структурується у декілька розділів, якщо він містить відомості 
різного плану або під час звіту виконувалось декілька задач (тоді кожен 
розділ відповідає окремій задачі).

\section{(Назва першого підрозділу)}

\emph{Наведені далі відомості у підрозділах розділів 1 та 2 стосуються 
виконання бакалаврської роботи, а не звіту. Сенсею просто ліньки це все 
переписувати.}

\emph{Відомості у вступах та висновках до розділів, а також у загальному вступі 
та загальних висновках є важливими.}

Перший розділ повинен бути присвячений огляду попередніх результатів за 
тематикою вашого дослідження. У даному розділі повинні міститись вс' 
визначення та описи, необхідні для подальшого викладення матеріалу, та результати 
ваших попередників.

Зауважимо, що наводити детальні доведення не ваших результатів необхідно 
наводити лише тоді, коли вони містять якусь вкрай важливу інформацію для 
саме ваших результатів.

Також зауважимо, що абсолютно на всі не ваші результати повинні стояти 
належним чином оформлені посилання.

Розмір першого (оглядового) розділу не повинен перевищувати третини вашої 
дипломної роботи (без урахування додатків).


\section{(Назва другого підрозділу)}

Наведемо основні правила оформлення текстів у системі \LaTeX.

Для абзацу робіть пусті рядки у файлі. Курсивний текст робиться командою 
emph: \emph{ось так}. Жирний текст робиться командою textbf: \textbf{ось так}.

<<Лапки>> робляться двома знаками більше та двома знаками менше. Довге 
тире у тексті --- трьома дефісами, коротке -- двома дефісами; у формулах 
мінуси робляться одним дефісом: $a-b$.

Пишіть звичайний текст звичайним текстом, а формули, позначення змінних та 
операцій (усі формули, усі позначення змінних та усі операції) беріть у 
знаки долара, ось так: $E = mc^2$, $a_1 = a^{(2)} \cdot a_{n, k}$, $e^x = 
\sum_{k = 0}^{\infty} {\frac{x^k}{k!}}$. Якщо вам 
не подобається, як \LaTeX подав формулу для експоненти (мені, наприклад, 
не подобається), то можна внести у код формули деякі корективи та написати ось так: $e^x 
= \sum\limits_{k = 0}^{\infty} {\dfrac{x^k}{k!}}$.

Виключна формула (формула окремим рядком) робиться через подвійні знаки 
долара або через оточення equation. Зауважте, що при цьому змінюється 
оформлення формул:
$$e^x = \sum_{k = 0}^{\infty} {\frac{x^k}{k!}}.$$

Формули за помовчанням не підтримують кирилічні літери. Зверніть увагу на 
порожній рядок перед попереднім реченням у tex-файлі: без нього не буде 
створено абзац.

Із більш специфічних позначень --- ось так, скажімо, можна подати 
перестановку:
$$\pi = \begin{pmatrix}
1 & 2 & 3 & 4 & 5 & 6 & 7 & 8 & 9\\
a & 5 & 9 & 6 & 4 & 8 & 2 & 1 & 7
\end{pmatrix},$$
де $a=3$. Зауважте, що у попередньому реченні нема порожнього рядочку 
перед <<де>> (та, відповідно, абзацу після формули), а кома внесена у 
виключну формулу, бо інакше вона переїде у наступний рядок тексту.

Декілька формул поспіль треба збирати в єдине ціле оточеннями gather або 
eqnarray; назви оточень із зірочками вказують \LaTeX'у не нумерувати дані 
формули. Наприклад, ось рекуренти для циклових чисел та чисел Стірлінга 
I~роду:
\begin{eqnarray*}
c(n+1, k) &=& c(n, k-1)+nc(n, k); \\
s(n+1, k) &=& s(n, k-1)-ns(n, k).
\end{eqnarray*}

Зверніть увагу на символ <<\verb|~|>> у попередньому абзаці tex-файлу між 
<<I>> та <<роду>>; це нерозривний пробіл, який не дасть рознести пов'язані 
частини по різних рядках. Тільду треба ставити перед усіма посиланнями 
(команди ref та cite), перед тире та у місцях, які не можна розривати за 
правилами граматики.

Для специфічних позначень ви можете задавати власні команди (їх 
рекомендовано заносити у файл <<\verb|02_redefinitions|>>). Наприклад, 
подивіться, як оформлюється теорема Лагранжа-Бюрмана із використанням 
введених команд \verb|\Coef| та \verb|\compinv|:

\begin{theorem}[Лагранж, Бюрман] \label{thLagrangeBurmann}
Для будь-якого ряду $A \in x \mathcal R[[x]]_1$ та $k \in \mathbb N$ справедливе співвідношення
$$n \Coef[x^n] \left( \compinv{A}(x) \right)^k = k \Coef[x^{n-k}] \left(\! \frac{x}{A(x)} \!\right)^n.$$
\end{theorem}
\begin{proof}
Доведення ви подивитесь деінде, а тут подивіться, як воно оформлюється 
(зокрема, на квадратик наприкінці :)).                                                                                       
\end{proof}

\begin{corollary} \par %\label{pr##}
Будь-ласка, перевіряйте граматику. Латеховські редактори зазвичай не мають 
інтегрованих спелчекерів української мови, тому використовуйте сервіси, 
наведені, наприклад, тут: https://coma.in.ua/30584
\end{corollary}

Іноді написаний файл треба компілювати двічі для одержання ефекту 
(скажімо, для коректної побудови усіх гіперпосилань та побудови змісту). 
Скажімо, оце посилання на теорему~\ref{thLagrangeBurmann} (теорему 
Лагранжа-Бюрмана) з першої компіляції може показати вам знаки питання 
<<??>>. Однак після повторної компіляції ви одержите те, що потрібно.

Онлайн-сервіси на кшталт Overleaf справляються з такими ситуаціями за одну компіляцію. Однак той 
же Overleaf має звичку компілювати pdf-файли навіть за наявності помилок у 
тексті, просто ігноруючи відповідні місця. Якщо ви працюєте у Overleaf, 
то переконайтесь, що у вас нема червоних помилок після компіляції.

Якщо вам потрібна якась фіча, запитайте в Сенсея. Майже напевно вона є.


\section{(Назва третього підрозділу)}


Надамо деякі рекомендації щодо використання даного стильового файлу.

\begin{theorem}
Використовуйте оточення \emph{theorem} для теорем.
\end{theorem}
\begin{proof}
Для доведень використовуйте оточення \emph{proof}.
\end{proof}
\begin{theorem}
Нумерація відбувається автоматично
\end{theorem}
\begin{claim}
Використовуйте оточення \emph{claim} для тверджень.
\end{claim}
\begin{lemma}
Використовуйте оточення \emph{lemma} для лем.
\end{lemma}
\begin{corollary}
Використовуйте оточення \emph{corollary} для наслідків.
\end{corollary}
\begin{definition}
Використовуйте оточення \emph{definition} для визначень.
\end{definition}
\begin{example}
Використовуйте оточення \emph{example} для прикладів, на які є посилання.
\end{example}
\begin{remark}
Використовуйте оточення \emph{remark} для зауважень. Зверніть увагу, як 
веде себе команда \textbf{emph}
\end{remark}


\chapconclude{\ref{chap:review}}

Наприкінці кожного розділу ви повинні навести коротенькі підсумки по його 
результатах. Зокрема, для оглядового розділу в якості висновків необхідно 
зазначити, які задачі у даній тематиці вже були розв'язані, а саме 
поставлена вами задача розв'язана не була (або розв'язана погано), тому у 
наступних розділах ви її й розв'язуєте.

Якщо ваш звіт складається з одного розділу, пропускайте висновок до 
нього~-- він повністю включається в загальні висновки до роботи