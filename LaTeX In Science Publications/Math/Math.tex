% !TeX program = lualatex
% !TeX encoding = utf8

\documentclass[14pt]{extarticle}

\usepackage{fontspec}
% \setmainfont{CMU Serif}
\setmainfont{Times New Roman}
\usepackage[english, ukrainian]{babel}

\usepackage{geometry} 
\geometry{left=2cm}
\geometry{right=2cm}
\geometry{top=2cm}
\geometry{bottom=2cm}

\usepackage{enumitem}
\usepackage{lipsum}

% \usepackage{tabularray}

\usepackage{xcolor}
\pagecolor[rgb]{0.118,0.118,0.118} % VS Code default dark background
\color[rgb]{0.8,0.8,0.8} % VS Code default dark text color

\usepackage{amsmath, amssymb}

% \NewDocumentCommand: https://www.texdev.net/2010/05/23/from-newcommand-to-newdocumentcommand/
\NewDocumentCommand\NoArguments{}{Command with no arguments, just simple text to insert.}
\NewDocumentCommand\Addition{mmm}{#1 plus #2 equals #3}
\NewDocumentCommand\Circle{mmO{R}}{$#1^2 + #2^2 = #3^2$}

\begin{document}

I'm here! ґ ї The simplest type of macro is one with no arguments at all. This isn’t going to show off xparse very much but is a starting point. The traditional method to do this is: 

\begin{enumerate}[label=($\bigstar$)]
    \item \NoArguments
    \item Let us add some numbers: \Addition{1}{2}{3}, and then \Addition{5}{6}{11}.
    \item Another example: \Circle{a}{b}[c] and \Circle{x}{y} are the same.
\end{enumerate}

Я пишу в \LaTeX{} різноманітні формули, наприклад: \Circle{x}{y}. Пізніше я ще напишу купу всякого. Ось, наприклад формула між абзацами:
\begin{equation}\label{eq: circle equation}
    x^2 + y^2 = R^2
\end{equation}

Наостанок, формула \eqref{eq: circle equation} -- \& це рівняння кола.Захищаючи Україну зараз, ми па\-м'ятаємо, що наша міцність і сила духу базуються на міцності й силі багатьох українців, які не здавалися, які мріяли та діяли заради того, щоб Україна жила. Самостійна, вільна і сильна Україна, європейська і демократична, незалежна і цілісна.

Ajhvekf $ x^2$ nf \(x^2\). 
$$ x^2 $$ \[ x^2 \] 


$$ [ \frac{1}{2} ] $$
$$ \left. \frac{x}{2}\, \right|_{0}^{12} $$

\[
    \begin{aligned}
        Q_m &
            = P(\eta > \xi_1 + \xi_2 + \ldots + \xi_m) = \int\limits_0^\infty P(\eta > \zeta_m)f_{\zeta_m}(u)du = 
            \int\limits_0^\infty \Bigl[ 1-F_{\zeta_m}(u) \Bigr]f_{\zeta_m}(u)du = \\
        & = \int\limits_0^\infty \frac{1}{1+u} \cdot \frac{u^{m-1}}{(m-1)!} \cdot e^{-u} du 
          = \frac{1}{m-1} \int\limits_0^\infty \frac{u}{1+u} \cdot \frac{u^{m-2}}{(m-2)!} \cdot e^{-u} du = \\
        & = \frac{1}{m-1} \int\limits_0^\infty \frac{u}{1+u}\ f_{\zeta_{m-1}}(u) du
          = \frac{1}{m-1} \cdot M\frac{\xi_1 + \xi_2 + \ldots + \xi_{m-1}}{1+\xi_1 + \xi_2 + \ldots + \xi_{m-1}} \\
        & \widehat{q} = \frac{1}{m-1} \cdot \frac{\xi_1 + \xi_2 + \ldots + \xi_{m-1}}{1+\xi_1 + \xi_2 + \ldots + \xi_{m-1}} \\
    \end{aligned}
\]

\end{document}